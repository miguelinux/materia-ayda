% ex: ts=2 sw=2 sts=2 et filetype=tex
% SPDX-License-Identifier: CC-BY-SA-4.0
\begin{frame}
    \frametitle{Contenido}
    \tableofcontents
\end{frame}

\section{Problema}

\begin{frame}[c]{Problema de asignaciones estables}
  \begin{description}
    \item[Stable Matching Problem] David Gale \& Lloyd Shapley, 1962
  \end{description}

  \vspace{\baselineskip}
  Cómo diseñar el proces de admición de alumnos de la Universidad o de
  asignación de personal en una \textit{empresa}
  (p.ej. prácticas en empresa) ...

  \vspace{\baselineskip}
  ... teniendo en cuenta el orden de preferencia de las empresas y de los
  alumnos.
\end{frame}

\begin{frame}[c]{Problema de asignaciones estables}
  \begin{description}
    \item[Objetivo] Dado el conjunto de preferencia de las empresas y de los
      alumnos en prácticas, diseñar un proceso de asignación que conduzca a una
      \textbf{asignación estable}
    \vspace{\baselineskip}
    \item[Par inestable] El alumno $X$ y la empresa $Y$ son un par inestable si:
      \begin{itemize}
        \item $X$ prefiere $Y$ en vez de la empres a la que ha sido asignado.
        \item $Y$ prefiere a $X$ en lugar del alumno que le ha asignado.
      \end{itemize}
  \end{description}
\end{frame}

\begin{frame}[c]{Problema de asignaciones estables}
  \begin{description}
    \item[Asignación estable] Una asignación sin pares inestables.
  \end{description}

  \vspace{\baselineskip}
  \begin{block}{¿Por qué es deseable una asignación estable?}
    Porque se reduce la insatisfacción de empresas y alumnos, además de eliminar
    la necesidad de realizar resignaciones (nadie tiene incentivos para cambiar
    su asignación).
  \end{block}
\end{frame}

\begin{frame}[c]{En otros términos}
  \begin{block}{Problema del matrimonio estable}
    Es el problema de encontrar un emparejamiento estable entre dos conjuntos
    de elementos de igual tamaño dado un orden de preferencias para cada
    elemento. Una coincidencia es una biyección de los elementos de un
    conjunto a los elementos del otro conjunto.
  \end{block}
\end{frame}

\begin{frame}[c]{Problema del matrimonio estable}
  Dados $n$ hombre y $m$ mujeres, encontrar un emparejamiento estable de
  hombres con mujeres.

  \vspace{\baselineskip}
  Cada persona "evalúa" a las personas de sexo opuesto.
  \begin{itemize}
    \item Los hombres ordenan a las mujeres según sus preferencias.
    \item Las mujeres ordenan a los hombres según sus preferencias.
  \end{itemize}
\end{frame}

\begin{frame}[c]{Problema del matrimonio estable}
  \begin{block}{Emparejamiento perfecto}
    Todo el mundo emparejado de forma "monógama".
    \begin{itemize}
      \item Cada hombre emparejado con una mujer.
      \item Cada mujer emparejado con un hombre.
    \end{itemize}
  \end{block}
\end{frame}

\begin{frame}[c]{Problema del matrimonio estable}
  \begin{block}{Estabilidad}
    Ninguna pareja participante tiene incentivos para romper la asignación.
    \begin{itemize}
      \item Una pareja no emparejada $h-m$ si tanto el hombre $h$ como la mujer
        $m$ se prefieren mutuamente antes que a sus parejas actuales.
      \item La pareja inestable podría "mejorar" engañando a sus actuales
        parejas.
    \end{itemize}
  \end{block}
\end{frame}
