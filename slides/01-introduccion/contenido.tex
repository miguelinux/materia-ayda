% ex: ts=2 sw=2 sts=2 et filetype=tex
% SPDX-License-Identifier: CC-BY-SA-4.0
\begin{frame}
    \frametitle{Contenido}
    \tableofcontents
\end{frame}

\section{Concepto de algoritmo}

\begin{frame}[c]{Definición de algoritmo}
  \begin{block}{Definición}
    \begin{itemize}
      \item Conjunto ordenado y finito de operaciones que permite hallar
            la solución de un problema\footnote{Real academia española}.
      \item Conjunto de pasos, acciones o instrucciones necesarios para
            lograr un resultado o resolver un problema.
      \item Secuencia ordenada de pasos exentos de ambigüedad tal que, al
            llevarse a cabo con fidelidad, dará como resultado que se realice
            la tarea para la que se ha diseñado en un tiempo finito.
    \end{itemize}
  \end{block}
\end{frame}

\begin{frame}[c]{Propiedades de un algoritmo}
  \begin{description}
    \item[Finitud] La ejecución de un algoritmo ha de terminar después de un
      número finito de etapas.
    \item[Precisión] Cada etapa ha de estar especificado rigurosamente. La
      ejecución de un algoritmo no ha de dejar espacio para la interpretación,
      la intuición o la creatividad.
    \item[Ordenado] Las instrucciones se ejecutan una después de otra en un
      orden específico. Al cambiar el orden, puede cambiar el resultado.
    \item[Definido] Cada instrucción atiende un solo problema particular.
      No se presta a ambigüedades (dobles significados).
  \end{description}
\end{frame}

\section{Resolución de problemas}

\begin{frame}[c]{Título}
    \begin{center}
        Texto
    \end{center}
\end{frame}

\section{Clasificaicón de problemas}

\begin{frame}[c]{Título}
    \begin{center}
        Texto
    \end{center}
\end{frame}

\section{Algorítmica}

\begin{frame}[c]{Título}
    \begin{center}
        Texto
    \end{center}
\end{frame}

\section{Análisis de la eficiencia de los algoritmos}

\begin{frame}[c]{Título}
    \begin{center}
        Texto
    \end{center}
\end{frame}

\section{}

\begin{frame}[c]{Título}
    \begin{center}
        Texto
    \end{center}
\end{frame}

\section{Técnicas de diseño de algoritmos}

\begin{frame}[c]{Título}
    \begin{center}
        Texto
    \end{center}
\end{frame}
