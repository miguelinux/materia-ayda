% ex: ts=2 sw=2 sts=2 et filetype=tex
% SPDX-License-Identifier: CC-BY-SA-4.0
\documentclass[aspectratio=169]{beamer}

\usetheme{Boadilla}

\setbeamertemplate{navigation symbols}{} % To remove the navigation symbols from the bottom of all slides

\graphicspath{{../img/}{../styles/tecmm/}}
% use the command "usepackage" instead of "usecolortheme" to use the path to styles
\usepackage{../styles/tecmm/beamercolorthemetecmm}

\usepackage[utf8]{inputenc}
\usepackage[T1]{fontenc} %% https://tex.stackexchange.com/questions/664/why-should-i-use-usepackaget1fontenc
\usepackage{graphicx} % Allows including images
\usepackage{listings}

\title{La eficiencia de los algoritmos}

%\author[Miguel Bernal Marin]{\includegraphics[height=2.5cm]{tecmm-m-logo.png}\\ Miguel Bernal Marin} % Your name
\author{Miguel Bernal Marin} % Your name
\institute[TecMM Zapopan] % Your institution as it will appear on the bottom of every slide, may be shorthand to save space
{
 Instituto Tecnológico\\
 José Mario Molina\\
 Pasquel y Henríquez\\% Your institution for the title page
\medskip
\textit{miguel.bernal@zapopan.tecmm.edu.mx} % Your email address
}
\date{
    \today
} % Date, can be changed to a custom date

%New colors defined below
\definecolor{codeBakground}{rgb}{0.95,0.95,0.92}
\definecolor{codeComment}{rgb}{0.0, 0.0, 1.0}
\definecolor{codeKeyword}{rgb}{0.647, 0.165, 0.165}
\definecolor{codeKeyword2}{rgb}{0.0, 0.545,0.545}
\definecolor{codeNumbers}{rgb}{0.5,0.5,0.5}
\definecolor{codeString}{rgb}{1.0, 0.0, 1.0}
\definecolor{light-gray}{gray}{0.90}

%Code listing style named "mystyle"
\lstdefinestyle{mystyle}{
  backgroundcolor=\color{codeBakground},
  commentstyle=\color{codeComment},
  keywordstyle=\color{codeKeyword},
  keywordstyle={[2]\color{codeKeyword2}}, % Built-in
  numberstyle=\tiny\color{codeNumbers},
  stringstyle=\color{codeString},
  basicstyle=\ttfamily\footnotesize,
  breakatwhitespace=false,
  breaklines=true,
  captionpos=b,
  keepspaces=true,
  numbers=left,
  numbersep=5pt,
  showspaces=false,
  showstringspaces=false,
  showtabs=false,
  upquote=true,                      % requires textcomp
  tabsize=2
}

%"mystyle" code listing set
\lstset{style=mystyle}

\logo{\includegraphics[height=1.5cm]{tecmm-z-logo.png}}
\newcommand{\nologo}{\setbeamertemplate{logo}{}} % command to set the logo to nothing

%------------------------------------------------------------
%The next block of commands puts the table of contents at the 
%beginning of each section and highlights the current section:
\AtBeginSection[]
{
  \begin{frame}
    \frametitle{Contenido}
    \tableofcontents[currentsection]
  \end{frame}
}
%------------------------------------------------------------

\newcommand{\pausa}{\pause} % Para usar una pausa en las presentaciones
%\newcommand{\pausa}{}      % Para que NO salgan las pausas

\begin{document}

%%{\nologo
\begin{frame}
    \titlepage
\end{frame}
%%}

\begin{frame}
    \frametitle{Contenido}
    \tableofcontents
\end{frame}

% ex: ts=2 sw=2 sts=2 et filetype=tex
% SPDX-License-Identifier: CC-BY-SA-4.0
\begin{frame}
    \frametitle{Contenido}
    \tableofcontents
\end{frame}

\section{Concepto de algoritmo}

\begin{frame}[c]{Definición de algoritmo}
  \begin{block}{Definición}
    \begin{itemize}
      \item Conjunto ordenado y finito de operaciones que permite hallar
            la solución de un problema\footnote{Real academia española}.
      \item Conjunto de pasos, acciones o instrucciones necesarios para
            lograr un resultado o resolver un problema.
      \item Secuencia ordenada de pasos exentos de ambigüedad tal que, al
            llevarse a cabo con fidelidad, dará como resultado que se realice
            la tarea para la que se ha diseñado en un tiempo finito.
    \end{itemize}
  \end{block}
\end{frame}

\begin{frame}[c]{Propiedades de un algoritmo}
  \begin{description}
    \item[Finitud] La ejecución de un algoritmo ha de terminar después de un
      número finito de etapas.
    \item[Precisión] Cada etapa ha de estar especificado rigurosamente. La
      ejecución de un algoritmo no ha de dejar espacio para la interpretación,
      la intuición o la creatividad.
    \item[Ordenado] Las instrucciones se ejecutan una después de otra en un
      orden específico. Al cambiar el orden, puede cambiar el resultado.
    \item[Definido] Cada instrucción atiende un solo problema particular.
      No se presta a ambigüedades (dobles significados).
  \end{description}
\end{frame}

\section{Resolución de problemas}

\begin{frame}[c]{Título}
    \begin{center}
        Texto
    \end{center}
\end{frame}

\section{Clasificaicón de problemas}

\begin{frame}[c]{Título}
    \begin{center}
        Texto
    \end{center}
\end{frame}

\section{Algorítmica}

\begin{frame}[c]{Título}
    \begin{center}
        Texto
    \end{center}
\end{frame}

\section{Análisis de la eficiencia de los algoritmos}

\begin{frame}[c]{Título}
    \begin{center}
        Texto
    \end{center}
\end{frame}

\section{}

\begin{frame}[c]{Título}
    \begin{center}
        Texto
    \end{center}
\end{frame}

\section{Técnicas de diseño de algoritmos}

\begin{frame}[c]{Título}
    \begin{center}
        Texto
    \end{center}
\end{frame}


\end{document}
