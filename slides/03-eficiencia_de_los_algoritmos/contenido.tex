% ex: ts=2 sw=2 sts=2 et filetype=tex
% SPDX-License-Identifier: CC-BY-SA-4.0

\section{Comparación de algoritmos}

\begin{frame}[c]{Comparación de algoritmos}
  A menudo dispondremos de más de un algoritmo para
  resolver un problema dado, ¿con cuál nos quedamos?

  \vspace{\baselineskip}
  \textbf{Uso de recursos}

  \vspace{\baselineskip}
  \begin{itemize}
    \item Computacionales:
          \begin{itemize}
            \item Tiempo de ejecución
            \item Espacio de memoria
          \end{itemize}
    \item No computacionales:
          \begin{itemize}
            \item Esfuerzo de desarrollo (análisis, diseño e implementación)
          \end{itemize}
  \end{itemize}
\end{frame}

\begin{frame}[c]{Factores que influyen en el uso de recursos}
  \begin{itemize}
    \item Recursos computaciones:
          \begin{itemize}
            \item Diseño del algoritmo
            \item Complejidad del problema (p.ej. tamaño de las entradas)
            \item Hardware (arquitectura, MHz, MBs...)
            \item Calidad del código
            \item Calidad del compilador (optimizaciones)
          \end{itemize}
    \item Recursos no computaciones:
          \begin{itemize}
            \item Complejidad del algoritmo
            \item Disponibilidad de biblioteca reutilizables
          \end{itemize}
  \end{itemize}
\end{frame}

\section{Principio de invarianza}

\begin{frame}[c]{Principio de invarianza}
  \begin{itemize}
    \item Dos implementaciones de un mismo algoritmo no deferirán más
      que en una constante multiplicativa.
    \item Si $t_1(n)$ y $t_2(n)$ son los tiempos de dos implementaciones
      de un mismo algoritmo, se pude comprobar que:

      \vspace{\baselineskip}
      $\exists c,d \in \mathbb{R}, t_1(n) \leq ct_2()n); t_2(n) \leq dt_1(n)$
  \end{itemize}
\end{frame}
